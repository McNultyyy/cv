\documentclass[10pt,a4paper]{moderncv}
\moderncvtheme[blue]{classic}                
\usepackage[utf8]{inputenc}
\usepackage[scale=0.8]{geometry}
\usepackage{booktabs}
\usepackage{graphicx}
\usepackage{multirow}
\usepackage{tabularx}

\setlength\arrayrulewidth{.4pt}
\setlength\tabcolsep{6pt}

\firstname{William}
\familyname{McNulty}
\title{Software Developer}
\address{267a Ivydale Road}{Nunhead, London, SE15 3DZ}    
\mobile{07563 201 425}                    
\email{william.mcnulty@live.co.uk}         

\begin{document}
\maketitle

\section{Education}
\cventry
{Sept 2005\\June 2010}
{Kingsdale Foundation School}
{GCSE}
{}
{}
{8 GCSEs including Maths, English, IT and Physics\\*}

\cventry
{Sept 2010\\June 2012}
{Kingsdale Sixth Form}
{A-Levels}
{}
{}
{Maths, IT, Physics\\*}

\cventry
{Sept 2012\\May 2015}{Royal Holloway University of London}
{BSc. First Class Honours}
{}
{}
{Computer Science}

\hfill
\break

\section{Projects}
\cventry{Year 1\\Term 1}{Robotics}{}{}{}
{
	Our robotics group project involved building a rambler type robot that would use its sensors and Lejos NXT to navigate around obstacles as well as avoiding table edges and respond to sounds.
	\begin{itemize}
		\item Implemented behaviours using object oriented design. Storing these behaviours in an array and using an arbitrator to control them.
		\item Improved both the physical design and the internal logic of the robot over time.
		\item Wrote an extensive report containing annotated code, design features, and explaining how we solved problems that we had along the project.
		\item Presented our design as well as its features to our peers and lecturers.
	\end{itemize}
}
\cventry{Year 1\\Term 2}{XNA Games}{}{}{}
{
	My XNA project involved me design and create the game Asteroids, whilst including our own functionality and features.
	\begin{itemize}
		\item Created and tested code for the game using C\# in Microsoft XNA IDE. Having experienced similar style of coding in Java, C\# was quickly adopted.
		\item Produced a detailed report which explained the challenges we faced, the features we decided to implement, and how we achieved our goals.
		\item Designed and evaluated all aspects of the game as a group, analysing different approaches, suggestions and assessing ideas.
		\item Gained number of skills working in a team, listening and understanding ideas of others, accepting and taking on board constructive criticism, treating others with respect and keeping up group moral, being punctual and adjusting to different schedules and by fully immersing myself in the task at hand.
	\end{itemize}
}
\cventry{Year 2\\Term 1}{Software Engineering}{}{}{}
{
	This was an individual project in which I had to create a calculator with a graphical interface, using both reverse polish and infix notation.
	\begin{itemize}
		\item Learned how to efficiently create a project based on a given UML diagram.
		\item Effectively used source control to maintain work throughout the project.
		\item Used Agile methodologies such as TDD, use-cases, and user stories during the course of the project.
		item Gained the ability to create GUIs using WindowBuilder and involving it into the MVC pattern.
		\item Made sure my code complied against coding standards by adhering to a given Checkstyle rule set.
	\end{itemize}
}
\cventry{Year 2\\Term 2}{Team Project}{}{}{}
{
	We were assigned the task of creating an automated restaurant system in which members of staff and customers used a tablet to communicate. The project consisted of 5 sprints, each lasting 2 weeks, and we were constantly communication to the product owners (the lecturers).
	\begin{itemize}
		\item Learned how to utilise the scrum based agile methodology.
		\item Created user stories based on the product specification and turned them into tasks.
		\item Used the ProjectCards management tool throughout to assess to project progress.
		\item We held many scrum meetings each week to emphasize working as a team.
		\item Scrum master was changed each week which allowed me to understand how important communication and respect within a team is.
		\item Sprint review meetings were held at the end of each sprint and allowed us to demo the current working product (vertical and horizontal slices).
	\end{itemize}
}
\cventry{Year 3\\Term 1 \& 2}{Final Year Project}{}{}{}
{
	I chose to do my final year project on the subject of Value at Risk, which focuses on risk management of financial assets.
	\begin{itemize}
		\item Implemented different approaches of Value at Risk calculation, including Historical Simulation, Model Building and Monte Carlo Simulation.
		\item Analysed the accuracy of my implementation of each approach using backtesting and stress-testing.
		\item Allowed financial derivatives to be part of the portfolio.
		\item Gave a presentation of my program to my peers and project supervisors.
	\end{itemize}
}

\section{Work Experience}
\cventry{Summer 2014}
{Nursery Book}
{Junior Web Developer}
{}
{}
{
	I worked for a small start-up company called NurseryBook which aims to allow parents to view their child's development whilst in nursery through an on-line portal.
	\begin{itemize}
		\item Created CRUD style user interfaces which allowed the users to manage their records.
		\item Allowed users to view the progress of their nursery through the use of customised reports.
		\item Helped merge the back-end of the database with both the user interface on the website and the Android application.
	\end{itemize}
}

\cventry{Augest 2015 - Present}
{ICBC Standard Bank Plc}{Software Developer}{}{}
{
	I am currently working at ICBC Standard Bank Plc as a software developer with experience in both front and back-end development.
	\begin{itemize}
		\item Worked as part of a small team to create an in-house incident reporting application, known as ORIS, which allowed users to raise incidents following business workflows.\\
		As part of the project requirements, many different reports were built to satisfy business needs.\\
		An interactive dashboard was also created granted managers a high level view of the current state of the application.\\
		

}




\section{Technical Skills}
\cvcomputer{Database}
{MySQL, SQL Server}
{}
{}

\cvcomputer{Languages}
{C\#, Java, Javascript ES2016}
{}
{}

\cvcomputer{Frameworks}
{ASP.NET MVC, EntityFramework, Spring MVC}
{}
{}

\cvcomputer{Other}
{LaTeX, SSRS}
{}
{}

\section{References}
\qquad \qquad \qquad \quad Available on request

\end{document}
